% !Mode:: "TeX:UTF-8"

\chnunumer{10532}
\chnuname{湖南大学}
\cclassnumber{TP391}
\cnumber{S12102049}
\csecret{普通}
\cmajor{大规模图计算}
\cheading{硕士学位论文}      % 设置正文的页眉,以及自己的学位级别
\ctitle{基于单机多核系统的图处理研究}  %封面用论文标题,自己可手动断行
\etitle{Graph Processing Research on Single Multi-core Systems}
\caffil{信息科学与工程学院} %学院名称
\csubjecttitle{学科专业}
\csubject{计算机科学与技术}   %专业
\cauthortitle{研究生}     % 学位
\cauthor{周东伟}   %学生姓名
\ename{Dongwei Zhou}
\cbe{B.E.~(Northwestern Polytechnical University)~2012}
%\cms{M.S.~(Hunan University)2010}
\cdegree{thesis}
\cclass{Master of engineering}
\emajor{Computer Science and Technology}
\ehnu{Hunan~University}
\esupervisor{Hao Chen}
\csupervisortitle{指导教师}
\csupervisor{陈浩~教授} %导师姓名
\cchair{王东}
\ddate{2015年5月26日}
\edate{May,~2015}

\untitle{湖~~南~~大~~学}
\declaretitle{学位论文原创性声明}
\declarecontent{
本人郑重声明:所呈交的论文是本人在导师的指导下独立进行研究所取得的研究成果。除了文中特别加以标注引用的内容外,本论文不包含任何其他个人或集体已经发表或撰写的成果作品。对本文的研究做出重要贡献的个人和集体,均已在文中以明确方式标明。本人完全意识到本声明的法律后果由本人承担。
}
\authorizationtitle{学位论文版权使用授权书}
\authorizationcontent{
本学位论文作者完全了解学校有关保留、使用学位论文的规定,同意学校保留并向国家有关部门或机构送交论文的复印件和电子版,允许论文被查阅和借阅。本人授权湖南大学可以将本学位论文的全部或部分内容编入有关数据库进行检索,可以采用影印、缩印或扫描等复制手段保存和汇编本学位论文。
}
\authorizationadd{本学位论文属于}
\authorsigncap{作者签名:}
\supervisorsigncap{导师签名:}
\signdatecap{签字日期:}


%\cdate{\CJKdigits{\the\year} 年\CJKnumber{\the\month} 月 \CJKnumber{\the\day} 日}
% 如需改成二零一二年四月二十五日的格式,可以直接输入,即如下所示
% \cdate{二零一二年四月二十五日}
\cdate{2015年5月11日} % 此日期显示格式为阿拉伯数字 如2012年4月25日
\cabstract{

随着各种类型的社交网络的兴起,基于图结构数据的企业级应用正变得日益广泛与重要。而如何高效便捷的分析、调试和处理这些与日俱增的大规模图数据成为当前高性能计算领域的研究人员所面临的最迫切的问题之一。目前,已经存在一些解决方案,但是这些解决方案还存在着一定的问题。例如,分布式解决方案中存在着负载均衡、通信延迟以及经济成本高的问题,而单机系统的解决方案则存在着并发度低,容错性差等问题。

针对这些问题,本文从兼容性、容错性与便捷高效的角度考虑,提出了基于并行BSP(Bulk Synchronous Parallel,整体同步并行)模型的单机图处理系统GPSA(a Graph Processing System with Actors,基于Actor的图处理系统)。首先,GPSA使用Actor的并发模型改善系统的并发性和计算吞吐量。Actor模型不仅充分的利用多核的优势,同时避免频繁的上下文的切换所带来的性能损耗。

其次,改进传统的BSP计算模型。基于BSP模型的图处理过程主要有计算和消息分发两个步骤。在传统的BSP模型中,由于图数据的局部性问题,在以顶点为中心的实现方式中单个顶点上的计算和通信两个步骤需要顺序执行。GPSA结合Actor与BSP模型将图计算中顶点的计算和消息通信过程解耦,降低两个相邻的超级步之间的依赖关系,使计算和通信两个步骤并发执行,提升计算效率。

最后,在I/O优化方面,GPSA将图数据分为两个部分:顶点状态信息数据和边结构数据。其中,对于顶点的状态信息数据,GPSA利用内存映射将其映射到内存,从而提高数据的读取和更新能力。而边结构数据则保存在磁盘上顺序访问。实验证明,GPSA不仅能够显著提升单机系统上图处理的性能,同时还具有较好的容错性和灵活性。

}

\ckeywords{图处理;Actor模型;BSP模型}

\eabstract{
Graph-based applications become more and more common due to the rising of all kinds of online social networks. How to analyse, debug and develop an agile graph processing system becomes one of the most urgent problems facing systems researchers. Though some approaches have been proposed, however, these approaches still have some unresolved problems. For example, the distributed approaches have load balancing, communication latency and the cost issues. While the low degree of the concurrency and the poor fault-tolerant remains in the single machine approaches.

Motivated by this, in this paper, we introduce GPSA(a Graph Processing System with Actors), a single-machine graph processing system based on a parallel BSP(Bulk Synchronous Parallel) computation model with actors by considering the compatibility, convenience and fault-tolerant. First, GPSA takes advantage of actors to improve the concurrent degree and the throughput on a single machine, which could not only make full use of multi-core but also avoid the performance loss caused by the frequent context switching.

 Second, GPSA improves the BSP model. The traditional BSP based system has two main procedures: the computing and the message dispatching. Because of the locality of the graph processing, the two procedures in the vertex-centric implementation are executed sequentially.  GPSA improves the BSP computation model to fit actor programming model by decoupling the message dispatching procedure from computing procedure to remove the dependencies of two adjacent superstep and making the two procedures executed parallelly. 

At last,  GPSA optimizes the I/O by separating the graph into two parts: the vertex data and the edge data. For the vertex data, GPSA exploit memory mapping to improve the IO performance to avoid frequent data loading or unloading operation by mapping the vertex into memory and store the edge data in the disk which is accessed sequentially. We show, through experiments and theoretical analysis, processing large-scale graph on a single machine with GPSA could not only improves the performance but also gains fault-tolerant and flexibility.  
}

\ekeywords{ Graph processing;~~ Actor model;~~BSP model}

\makecover

\clearpage
