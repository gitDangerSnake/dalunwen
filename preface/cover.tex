% !Mode:: "TeX:UTF-8"

\chnunumer{10532}
\chnuname{湖南大学}
\cclassnumber{TP391}
\cnumber{S12102049}
\csecret{普通}
\cmajor{多核虚拟化}
\cheading{硕士学位论文}      % 设置正文的页眉,以及自己的学位级别
\ctitle{基于单机多核系统的图处理研究}  %封面用论文标题,自己可手动断行
\etitle{Graph Processing on Heterogeneous Multi-core Systems}
\caffil{信息科学与工程学院} %学院名称
\csubjecttitle{学科专业}
\csubject{计算机科学与技术}   %专业
\cauthortitle{研究生}     % 学位
\cauthor{周东伟}   %学生姓名
\ename{ZHOU~~DongWei}
\cbe{B.E.~(Hunan University)~2015}
%\cms{M.S.~(Hunan University)2010}
\cdegree{thesis}
\cclass{Master of engineering}
\emajor{Computer Science and Technology}
\ehnu{Hunan~University}
\esupervisor{Chen Hao}
\csupervisortitle{指导教师}
\csupervisor{陈浩~教授} %导师姓名
\cchair{~~~~~~~~}
\ddate{~~~~~~~~年~~~~月~~~~日}
\edate{April,~2014}

\untitle{湖~~南~~大~~学}
\declaretitle{学位论文原创性声明}
\declarecontent{
本人郑重声明:所呈交的论文是本人在导师的指导下独立进行研究所取得的研究成果。除了文中特别加以标注引用的内容外,本论文不包含任何其他个人或集体已经发表或撰写的成果作品。对本文的研究做出重要贡献的个人和集体,均已在文中以明确方式标明。本人完全意识到本声明的法律后果由本人承担。
}
\authorizationtitle{学位论文版权使用授权书}
\authorizationcontent{
本学位论文作者完全了解学校有关保留、使用学位论文的规定,同意学校保留并向国家有关部门或机构送交论文的复印件和电子版,允许论文被查阅和借阅。本人授权湖南大学可以将本学位论文的全部或部分内容编入有关数据库进行检索,可以采用影印、缩印或扫描等复制手段保存和汇编本学位论文。
}
\authorizationadd{本学位论文属于}
\authorsigncap{作者签名:}
\supervisorsigncap{导师签名:}
\signdatecap{签字日期:}


%\cdate{\CJKdigits{\the\year} 年\CJKnumber{\the\month} 月 \CJKnumber{\the\day} 日}
% 如需改成二零一二年四月二十五日的格式,可以直接输入,即如下所示
% \cdate{二零一二年四月二十五日}
\cdate{~~~~~~~~年~~~~月~~~~日} % 此日期显示格式为阿拉伯数字 如2012年4月25日
\cabstract{

随着各种类型的社交网络的兴起,基于图结构数据的企业级应用正变得日益广泛与重要。而如何高效便捷的分析、调试和处理这些与日俱增的大规模图数据成为当前高性能计算领域的研究人员所面临的最迫切的问题之一。目前,已经存在一些分布式的解决方案,但是基于分布式的图处理系统依然存在着许多悬而未决的难题。从经济角度来看,在分布式系统上进行数据分析、调试和处理需要额外的资源消耗,例如计算资源以及维持计算的能源消耗。与此同时,从用户角度分析,分布式系统对于相关的开发人员也提出了较高的要求,例如,开发经验,从而增加应用成本。而就分布式系统本身而言,不同计算节点之间的消息延迟与负载均衡则很容易成为系统的性能瓶颈。于是,有相关学者和研究人员提出基于单机共享内存的图处理系统,实验证明,这些单机上图处理系统经过严谨而合理的设计不仅具有高效便捷的优势,还能大大的降低开发成本。但是,随着网络的发展,数据量的海量递增已经越来越明显,单机系统在扩展性上很难满足这样的挑战。目前应用于企业级的图处理系统大部分仍然基于BSP模型的分布式系统,而现有单机系统往往从系统IO优化处着手,两者之间很难结合在一起。

本文从兼容性、容错性与便捷性的角度考虑,提出了基于并行BSP模型的单机图处理系统GPSA。基于BSP模型的图处理过程主要有计算和消息分发两个步骤。在传统的BSP模型中,由于图数据的局部性问题,计算和通讯两个步骤需要顺序执行。在本文,GPSA利用Actor编程模型取代线程来改善系统的并发性和计算吞吐量;其次,结合Actor与BSP模型将图计算中顶点的计算和消息处理的过程解耦,降低两个相邻的超级步之间的依赖关系,使计算和通讯两个步骤以流水线的方式并行执行。另外,GPSA利用内存映射的方式来提高数据的读取和更新能力。通过不同的数据集对比,不同的单机系统的对比,GPSA具有良好的图处理性能。

}

\ckeywords{图处理;Actor模型;BSP模型}

\eabstract{
Graph-based applications become more and more common due to the rising of all kinds of online social networks and other problems encountered in enterprise development environment. Due to the increasing need to process the fast growing graph-structured data (e.g., social networks and web graphs), analysing, debugging and developing an agile graph processing system becomes one of the most urgent problems facing systems researchers. Though some distributed approaches have been proposed, however, these approaches still have many unresolved problems. In the perspective of economy, analysing, debugging and processing large-scale graph needs the extra resource, such as computing resources and maintain calculation of energy consumption.  Furthermore, it requires the developer to be skilled such as rich experiences of distributed developing which increases the cost. In the perspective of the distributed approaches, the load balancing and the communication latency  among different computing nodes are possibly to be the bottleneck of the system. Therefore, some researchers proposed the approaches on just one PC. Experiments show that the single machine approaches with a rigorous and reasonable design could not only gain the reasonable performance but also can greatly reduce development costs. However, with the developing of the internet, the amount of the big data is becoming larger and larger, it is hardly to satisfy the challenge with the single machine approach. Nowadays the enterprise graph processing system is based on the BSP(Bulk Synchronous Parallel) model while the single machine approach mainly focus the IO performance, which makes the two hard to work together.

Motivated by this, in this paper, we introduce GPSA, a single-machine graph processing system based on a parallel BSP computation model by considering the compatibility, convenience and fault-tolerant. The traditional BSP based system has two main procedure: the computing and the dispatching. Because of the locality of the graph processing, the two procedure in the traditional BSP model is executed sequentially. In this paper, GPSA takes advantage of actors to improve the concurrent degree on a single machine with limited resource. GPSA improves the BSP computation model to fit actor programming model by decoupling the message dispatching procedure from computing procedure. Furthermore, we exploit memory mapping to improve the IO performance to avoiding frequent data loading or unloading operation. We show, through experiments and theoretical analysis, processing large-scale graph on a single machine with GPSA performs well.  
}

\ekeywords{ Graph processing;~~ Actor model;~~BSP model}

\makecover

\clearpage
