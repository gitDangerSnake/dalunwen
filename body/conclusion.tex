% !Mode:: "TeX:UTF-8"

\addcontentsline{toc}{chapter}{结\quad 论} %添加到目录中
\chapter*{结\quad 论}

本文在分析分布式图处理系统和单机图处理系统的基础上,对目前图处理系统中的一些缺陷进行讨论,并在传统BSP计算模型的基础上,利用Actor并发模型对BSP模型进行优化,提出了基于Actor并发模型的BSP计算模型。通过对Actor-BSP模型的数据行为进行分析和讨论,将图数据分为两个部分:常驻内存的顶点状态信息和保存在磁盘上的边数据信息,同时利用内存映射优化IO操作。实验证明,GPSA图处理系统不仅能够在单机多核系统上高效的处理大规模图,同时还能够充分发挥多核的优势。本文的主要工作包括以下几个方面:

1、本论文在做了大量调研的基础上,对目前大规模图处理系统进行分析和对比,详细讨论目前存在于分布式图处理系统的困难问题,说明在单机系统上进行大规模图处理的可行性,对单机系统上的图处理系统进行分析和对比,从计算模型的角度出发,对传统的BSP模型进行优化,提出崭新的Actor-BSP图计算模型。Actor-BSP模型中将传统以顶点为中心的模型中顺序执行的计算过程和分发过程分离解耦,同时使用Actor代替线程,并将计算过程和分发过程分布在不同的Actor上,提高任务并发量。同时,Actor-BSP简化了图计算的流程,由于淡化顶点作为整个计算的中心的概念,顶点之间消息的传递转换为Actor之间的消息分发,计算过程和分发过程之间通过消息建立联系,从而无需再计算中缓存大量的消息,节省大量的IO操作。另外,Actor-BSP模型缩短了单个任务的平均执行时间,提升计算效率。

2、在Actor-BSP模型的基础上对数据的访问行为进行分析和讨论,由于新模型消息发送的随机性,就无法避免计算Actor对顶点状态信息的随机访问,与其他单机系统尽力回避随机访问的做法不同,GPSA大胆的采用数据分离的方法,将图数据分为两个独立的部分:顶点的状态信息和图的结构信息。其中,顶点的状态信息按照顺序存储的方式利用内存映射技术将其映射到内存来支持随机访问,提升效率。而对于图的结构信息则保持于磁盘上,可以采用顺序访问的方式,进一步节省IO操作。


3.独立完成了本文所论述的GPSA系统的调研,方案设计,具体编码实现和测试工作。列举PageRank、连通分量以及广度优先搜索三个常见图应用在GPSA系统上的实现,并将之与其他单机图处理系统从效率和多核利用率两个方面进行对比和分析,结果表明,GPSA不仅具有高效能够充分发挥多核优势的特点,而且具有较好的伸缩性与容错性。

虽然GPSA从改进计算模型角度出发,并取得良好的效果,但是GPSA依然存在一些不足,有待进一步完善。

首先,GPSA不支持图结构改变的应用。GPSA将图分为两部分:顶点信息和边。其中,顶点常驻内存,支持随机访问和更新,但是边、边上的权重等信息保存在磁盘上,需要尽量避免随机读写。

其次,GPSA使用JAVA实现,在实现过程中为避免频繁的垃圾回收造成的性能影响,在消息的封装中使用基本类型,造成消息的生成仅仅支持数字类型,无法支持字符串或者对象等。

最后,GPSA的设计初衷是能够兼容分布式以BSP为计算模型的大规模图处理系统,但是由于时间、经济等方面条件限制,该部分猜想的验证无力完成,希望能在将来能继续进行并开展在分布式方向的扩展和实验。
