% !Mode:: "TeX:UTF-8"

\chapter{GPSA系统设计与实现}

\section{计算模型简介}
大规模的图处理根据不同的计算平台会展现出不同的特性。在分布式或者云平台上,大规模图被分割分配在不同的的计算节点上,主要表现出计算密集性的特点。而单机的计算平台上,大规模图系统则同时表现出计算密集性和IO密集性两个特点。因此,在单机系统中设计大规模图处理系统的时候就需要同时兼顾两个特性。

由于大规模图的处理无法满足程序局部性的特征,很容易引起数据的随机访问,高效的处理大规模图变得非常困难。因此,针对大规模图处理有相关研究提出了全新的计算模型来适应大规模图的数据随机访问的特性。目前,已经存在几种比较成熟的计算模型,但是由于图计算模型之间并没有明确的区分标准,只能从同步或者异步角度粗略的分为两种:

\begin{itemize}
\item BSP计算模型
\item 异步计算模型
\end{itemize}

\section{BSP计算模型}
%http://blog.sina.com.cn/s/blog_8068adc50101cdnz.html

BSP模型最初作为一个并行计算领域中软件和硬件之间的“过渡模型”而提出的,是一个通用的并行计算模型,相对于MapReduce更适合构建大规模图处理原型系统中。BSP,即Bulk Synchronous Parallel,“大块”同步模型,其概念由哈佛大学的Valiant和牛津大学的Bill McColl提出,是一种异步MIMD-DM模型,支持消息传递系统,块内异步并行,块间显式同步。

在大规模图处理系统中,BSP模型的实现由可以进一步分为两大类:以顶点为中心和以边为中心。在以Pregel为代表的Vertex-Centric模型中,顶点为中心的计算模型和边为中心的计算模型。在以顶点为中心的计算模型中,由用户提供针对每个顶点上的处理函数,顶点之间互相通讯。整个计算过程由一系列的超级步组成,在每一个超级步内,一切处理围绕顶点展开,由顶点完成一系列同步计算。首先,顶点处理来自入边的更新消息,完成本超级步内的计算。然后,再将自己的更新消息发送给其邻接顶点。最后,所有顶点完成本超级步的处理之后,整个图进入下一个超级步,处理流程如图\ref{}所示。

X-Stream以边为中心的计算模型则是从边的视角出发,将计算组织成一系列的迭代过程,将计算过程分为两个发散和收集两个步骤。在发散阶段,X-Stream通过边将更新信息由源点发送到目的顶点,在收集阶段,则对边的目的顶点进行更新。然后,对所有边进行迭代循环处理,当所有边都完成处理之后,进入下一个超级步,处理流程如图\ref{}所示。

\section{异步计算模型}
目前,基于异步计算模型的图处理系统主要以GraphChi为代表。与BSP的同步计算中不同的是GraphChi默认最新的消息对于后续顶点总是可见。因此,GraphChi是一个基于磁盘的异步计算模型。GraphChi作为一个运行于单机系统上的图处理系统,为了解决随机访问所带来的问题,设计硬盘平行滑动窗口(PSW)来减少随机读写的。平行滑动窗口(PSW)包括若干Interval和Shard,其中Interval表示处理顶点的区间,Shard包含了目标顶点在当前Interval的边,并且这些边按源顶点顺序保存。严格来说,GraphChi遵守Vertex-Centric的模型的基本规则,以顶点为计算的中心病调用用户提供的处理函数。但是,在处理过程中,GraphChi的处理对各个Interval和Shard进行依次处理,已经处理过的Interval的顶点的状态对于之后处理的顶点而言是可见的,打破了BSP同步计算中顶点的状态需要同步之后,在下一轮迭代中进行计算所带来的开销。

\section{BSP模型的缺陷}
如图所示,BSP模型


\section{模型改进}


