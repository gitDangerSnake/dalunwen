% !Mode:: "TeX:UTF-8"

\chapter{大规模图计算模型}

\section{计算模型简介}

大规模图数据具有数据量大、类型繁多、价值密度低、速度快时效高等特性。


\section{BSP计算模型}

%http://blog.sina.com.cn/s/blog_8068adc50101cdnz.html

BSP模型最初作为一个并行计算领域中软件和硬件之间的“过渡模型”而提出的,是一个通用的并行计算模型,相对于MapReduce更适合构建大规模图处理原型系统中。BSP,即Bulk Synchronous Parallel,“大块”同步模型,其概念由哈佛大学的Valiant和牛津大学的Bill McColl提出,是一种异步MIMD-DM模型,支持消息传递系统,块内异步并行,块间显式同步。该模型基于一个Master协调,所有的Worker同步执行, 数据从输入的队列中读取,处理过程如图3.1所示。

\section{Vertex-Centric模型}

\section{Edge-Centric模型}

\section{基于Actor的BSP模型}


