% !Mode:: "TeX:UTF-8"

\chapter{GPSA系统设计与实现}

\section{计算模型简介}
大规模的图处理根据不同的计算平台会展现出不同的特性。在分布式或者云平台上,大规模图被分割分配在不同的的计算节点上,主要表现出计算密集性的特点。而单机的计算平台上,大规模图系统则同时表现出计算密集性和IO密集性两个特点。因此,在单机系统中设计大规模图处理系统的时候就需要同时兼顾两个特性。

由于大规模图的处理无法满足程序局部性的特征,很容易引起数据的随机访问,高效的处理大规模图变得非常困难。因此,针对大规模图处理有相关研究提出了全新的计算模型来适应大规模图的数据随机访问的特性。目前,已经存在几种比较成熟的计算模型,但是由于图计算模型之间并没有明确的区分标准,只能从宏观同步或者异步角度粗略的分为两种:

\begin{itemize}
\item BSP计算模型
\item 异步计算模型
\end{itemize}

\section{BSP计算模型}
%http://blog.sina.com.cn/s/blog_8068adc50101cdnz.html

BSP模型最初作为一个并行计算领域中软件和硬件之间的“过渡模型”而提出的,是一个通用的并行计算模型,相对于MapReduce更适合构建大规模图处理原型系统中。BSP,即Bulk Synchronous Parallel,“大块”同步模型,其概念由哈佛大学的Valiant和牛津大学的Bill McColl提出,是一种异步MIMD-DM模型,支持消息传递,块内异步并行,块间显式同步。

一个BSP模型的计算系统一般由三个部分组成:一组具有局部内存的处理单元;一个连接所有处理单元的数据通信网络;支持对所有处理单元进行全局路障同步的机制。客户端提交作业前,需要将源数据加载到Worker上,然后向Master提交作业并等待作业完成。Master收到作业启动通知后,向各个Worker发送通知,同步启动任务,并在接下来的时间里控制超步迭代。各Worker则负责具体执行客户端提交的作业,期间需要接收其它Worker发送的消息,进行本地计算处理,然后根据情况向其它Worker发送消息。

%BSP模型非常适合分布式的图计算环境。首先,BSP模型具有全局的数据通信网络。BSP中的各个处理单元和通过它和其他处理单元进行通信或内存的存取,这和大多数基于分布式内存或消息传递的并行模型是相同的。不同的是BSP的通信是一个全局的概念,不是点对点的通信,能够很好的解决图计算过程中数据的随机访问问题。其次,BSP模型具有全局的路障同步机制。路障同步的引入则能保证图处理过程的完整性,提高整个图处理的鲁棒性和可靠性。因此,很多基于分布式内存的大规模图处理统都构建在BSP模型的基础上,例如Pregel、GPS等。

在大规模图处理系统中,BSP模型的实现由可以进一步分为两大类:以顶点为中心和以边为中心。在以Pregel为代表的Vertex-Centric模型中,顶点为中心的计算模型和边为中心的计算模型。在以顶点为中心的计算模型中,由用户提供针对每个顶点上的处理函数,顶点之间互相通讯。整个计算过程由一系列的超级步组成,在每一个超级步内,一切处理围绕顶点展开,由顶点完成一系列同步计算。首先,顶点处理来自入边的更新消息,完成本超级步内的计算。然后,再将自己的更新消息发送给其邻接顶点。最后,所有顶点完成本超级步的处理之后,整个图进入下一个超级步,处理流程如图\ref{}所示。X-Stream以边为中心的计算模型则是从边的视角出发,将计算组织成一系列的迭代过程,将计算过程分为两个发散和收集两个步骤。在发散阶段,X-Stream通过边将更新信息由源点发送到目的顶点,在收集阶段,则对边的目的顶点进行更新。然后,对所有边进行迭代循环处理,当所有边都完成处理之后,进入下一个超级步,处理流程如图\ref{}所示。






\section{GraphChi异步计算模型}
目前,基于异步计算模型的图处理系统主要以GraphChi为代表。与BSP的同步计算中不同的是GraphChi默认最新的消息对于后续顶点总是可见。因此,GraphChi是一个基于磁盘的异步计算模型。GraphChi作为一个运行于单机系统上的图处理系统,为了解决随机访问所带来的问题,设计硬盘平行滑动窗口(PSW)来减少随机读写的。平行滑动窗口(PSW)包括若干Interval和Shard,其中Interval表示处理顶点的区间,Shard包含了目标顶点在当前Interval的边,并且这些边按源顶点顺序保存。严格来说,GraphChi遵守Vertex-Centric的模型的基本规则,以顶点为计算的中心病调用用户提供的处理函数。但是,在处理过程中,GraphChi的处理对各个Interval和Shard进行依次处理,已经处理过的Interval的顶点的状态对于之后处理的顶点而言是可见的,打破了BSP同步计算中顶点的状态需要同步之后,在下一轮迭代中进行计算所带来的开销。


\section{BSP模型的缺陷}

一个基于BSP程序同时具有水平和垂直两个方面的结构。从垂直上看,一个BSP程序由一系列串行的超步组成,每个超步又分为3个过程[12]:

(1)各个处理机进行本地局部计算。

(2)各个处理机利用本地内存中的信息完成局部的计算工作,在这一阶段处  理机发出远程内存读取和消息通信等工作。

(3)所有处理机进行全局路障同步,本次超步的通信操作在路障同步之后变为有效。


\section{计算模型改进}

BSP模型在分布式图处理系统中被广泛采用,主要是因为BSP模型非常适合分布式的图计算环境。首先,BSP模型具有全局的数据通信网络。BSP中的各个处理单元和通过它和其他处理单元进行通信或内存的存取,这和大多数基于分布式内存或消息传递的并行模型是相同的。不同的是BSP的通信是一个全局的概念,不是点对点的通信,能够很好的解决图计算过程中数据的随机访问问题。其次,BSP模型具有全局的路障同步机制。路障同步的引入则能保证图处理过程的完整性,提高整个图处理的鲁棒性和可靠性。因此,很多基于分布式内存的大规模图处理统都构建在BSP模型的基础上,例如Pregel、GPS等。

鉴于当前的计算机以多核为主,单个计算机的计算能力、并发处理能力已经有了很大的提高,将BSP模型从分布式的环境中迁移到这样的多核的计算机上不仅可以充分利用多核的计算资源,还能够降低图计算的处理成本。同时,考虑到单机图处理系统与分布式图处理系统的兼容性,本文首先对BSP模型中存在的问题进行了分析,然后针对这些问题提出New BSP模型,并在New BSP模型的基础上实现一个便捷、高效、可靠的单机图处理系统GPSA。

\vspace{\baselineskip}
\begin{figure}[htbp]
\centering
\begin{minipage}{0.4\textwidth}
\centering
\includegraphics[width=\textwidth]{myfigures/sequentialbsp_new}
\caption{传统BSP模型}\label{fig:traBSP}
\end{minipage}
\begin{minipage}{0.4\textwidth}
\centering
\includegraphics[width=\textwidth]{myfigures/computemodel}
\caption{改进后的BSP模型}\label{fig:newBSP}
\end{minipage}
\vspace{\baselineskip}
\end{figure}


\subsection{传统BSP模型的缺陷}
由于BSP模型的同步路障机制的存在,那么在整个BSP模型的垂直方向上,影响最终效率的关键就是最晚完成的任务,如图\ref{fig:bsp}所示。在基于传统BSP计算模型的图处理系统中,如图\ref{fig:traBSP}所示,图的处理主要分为两个阶段:以顶点为中心的计算(Compute)过程和消息的分发(Dispatch)过程。这两个过程按照严格的串行方式执行的,消息分发过程对计算过程存在数据依赖,形成强耦合。这种串行执行的方式,进一步延长了垂直方向上的任务完成的时间,降低的执行效率。在以顶点为中心的图计算中,消息时顶点之间的主要数据交换,顶点调用用户提供的计算函数对数据进行处理。消息包含计算所需的全部数据,所以相对于计算过程而言消息的具体分发过程是可以透明的。由于消息的生成和处理分布在两个相邻的超级步中,在该超级步结束之前,需要缓存大量的消息,并且在下一个超级步开始之前,这些消息不会被处理掉,增加额外的IO开销。
目前,基于BSP计算模型的图处理系统多任务并行处理的方法主要是采用多线程的技术。而线程是操作系统的基本调度单位,在操作系统中利用多线程在并发度上就会受到很大的限制。另外,由于需要保存大量的数据,当线程处理IO操作的时候,线程的调度会引发上下文的频繁切换,会对处理效率造成影响。

\subsection{Actor-BSP模型}

消息的分发过程和计算过程之间的主要依赖关系是消息,那么消息分发过程是消息的生产者,计算过程则是消息的消费者,两者之间以生产者和消费者的模式进行共存,从而将两者从严格串行的模式中解耦出来,如图\ref{fig:newBSP}所示。在改进后的BSP模型中,消息分发过程和计算过程位于两个单独的执行流程中。消息分发过程主要负责消息的生成和转发。计算过程侦听消息,当消息达到,计算过程则负责对消息进行处理及数据更新。
在语义上计算过程和消息分发过程是相互独立的轻量级调度单位。现有的BSP的实现在语义封装上往往是以顶点为中心,线程负责处理顶点。所以,在实现New BSP的模型中,采用轻量级、能够异步处理消息的并发模型成为关键。Actor并发模型相比于线程而言更加轻量级,有着更好的并发度。同时,Actor在语义上与Vertex的语义较为接近,兼顾了线程的调度执行和以顶点为中心的特点。

\subsection{New BSP模型的优势}

New BSP模型是在执行过程上完全异步计算的模型,计算过程与消息分发过程想分离,使得两个过程可以在一定程度上并行执行,充分利用多核的优势。首先,在New BSP模型中,Actor分为两种类型:负责消息分发过程的dispatching actor和负责计算过程的computing actor。得益于dispatching actor和computing actor之间的松耦合设计,两者之间的映射关系变得相当的灵活,缩短BSP模型在垂直方向上的执行流程,提高效率。
其次,dispatching actor 和 computing actor 组成了生产者和消费者模型,当消息到达computing actor之后,该actor会被调度执行,处理消息,从而无需保存这些消息,避免将消息进行缓存的IO开销。

\section{}


另外,在以顶点为中心的计算模型中,只有顶点的状态发生改变之后,才会发送消息,所以需要额外保存一份之前的状态数据来与当前计算过程完成之后的状态进行比较以判别顶点状态是否发生了改变。但是,在传统的BSP模型中,这两份状态数据,之前的状态数据直到下一个超步来临之前并不会被覆盖,并且这部分数据所占用的空间被白白浪费。

从BSP计算模型的垂直方向上看,它是由一些列串行的超步组成,而在一个超步的内部,图处理的两个过程也是串行处理。但是,在超级步S中
