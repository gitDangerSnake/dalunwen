% !Mode:: "TeX:UTF-8"

\chapter{GPSA系统设计与实现}

\section{计算模型简介}
大规模的图处理根据不同的计算平台会展现出不同的特性。在分布式或者云平台上,大规模图被分割分配在不同的的计算节点上,主要表现出计算密集性的特点。而单机的计算平台上,大规模图系统则同时表现出计算密集性和IO密集性两个特点。因此,在单机系统中设计大规模图处理系统的时候就需要同时兼顾两个特性。

由于大规模图的处理无法满足程序局部性的特征,很容易引起数据的随机访问,高效的处理大规模图变得非常困难。因此,针对大规模图处理有相关研究提出了全新的计算模型来适应大规模图的数据随机访问的特性。目前,已经存在几种比较成熟的计算模型,但是由于图计算模型之间并没有明确的区分标准,只能从宏观同步或者异步角度粗略的分为两种:

\begin{itemize}
\item BSP计算模型
\item 异步计算模型
\end{itemize}

\section{BSP计算模型}
%http://blog.sina.com.cn/s/blog_8068adc50101cdnz.html

BSP模型最初作为一个并行计算领域中软件和硬件之间的“过渡模型”而提出的,是一个通用的并行计算模型,相对于MapReduce更适合构建大规模图处理原型系统中。BSP,即Bulk Synchronous Parallel,“大块”同步模型,其概念由哈佛大学的Valiant和牛津大学的Bill McColl提出,是一种异步MIMD-DM模型,支持消息传递,块内异步并行,块间显式同步。

一个BSP模型的计算系统一般由三个部分组成:一组具有局部内存的处理单元;一个连接所有处理单元的数据通信网络;支持对所有处理单元进行全局路障同步的机制。客户端提交作业前,需要将源数据加载到Worker上,然后向Master提交作业并等待作业完成。Master收到作业启动通知后,向各个Worker发送通知,同步启动任务,并在接下来的时间里控制超步迭代。各Worker则负责具体执行客户端提交的作业,期间需要接收其它Worker发送的消息,进行本地计算处理,然后根据情况向其它Worker发送消息。

%BSP模型非常适合分布式的图计算环境。首先,BSP模型具有全局的数据通信网络。BSP中的各个处理单元和通过它和其他处理单元进行通信或内存的存取,这和大多数基于分布式内存或消息传递的并行模型是相同的。不同的是BSP的通信是一个全局的概念,不是点对点的通信,能够很好的解决图计算过程中数据的随机访问问题。其次,BSP模型具有全局的路障同步机制。路障同步的引入则能保证图处理过程的完整性,提高整个图处理的鲁棒性和可靠性。因此,很多基于分布式内存的大规模图处理统都构建在BSP模型的基础上,例如Pregel、GPS等。

在大规模图处理系统中,BSP模型的实现由可以进一步分为两大类:以顶点为中心和以边为中心。在以Pregel为代表的Vertex-Centric模型中,顶点为中心的计算模型和边为中心的计算模型。在以顶点为中心的计算模型中,由用户提供针对每个顶点上的处理函数,顶点之间互相通讯。整个计算过程由一系列的超级步组成,在每一个超级步内,一切处理围绕顶点展开,由顶点完成一系列同步计算。首先,顶点处理来自入边的更新消息,完成本超级步内的计算。然后,再将自己的更新消息发送给其邻接顶点。最后,所有顶点完成本超级步的处理之后,整个图进入下一个超级步,处理流程如图\ref{}所示。X-Stream以边为中心的计算模型则是从边的视角出发,将计算组织成一系列的迭代过程,将计算过程分为两个发散和收集两个步骤。在发散阶段,X-Stream通过边将更新信息由源点发送到目的顶点,在收集阶段,则对边的目的顶点进行更新。然后,对所有边进行迭代循环处理,当所有边都完成处理之后,进入下一个超级步,处理流程如图\ref{}所示。






\section{GraphChi异步计算模型}
目前,基于异步计算模型的图处理系统主要以GraphChi为代表。与BSP的同步计算中不同的是GraphChi默认最新的消息对于后续顶点总是可见。因此,GraphChi是一个基于磁盘的异步计算模型。GraphChi作为一个运行于单机系统上的图处理系统,为了解决随机访问所带来的问题,设计硬盘平行滑动窗口(PSW)来减少随机读写的。平行滑动窗口(PSW)包括若干Interval和Shard,其中Interval表示处理顶点的区间,Shard包含了目标顶点在当前Interval的边,并且这些边按源顶点顺序保存。严格来说,GraphChi遵守Vertex-Centric的模型的基本规则,以顶点为计算的中心病调用用户提供的处理函数。但是,在处理过程中,GraphChi的处理对各个Interval和Shard进行依次处理,已经处理过的Interval的顶点的状态对于之后处理的顶点而言是可见的,打破了BSP同步计算中顶点的状态需要同步之后,在下一轮迭代中进行计算所带来的开销。


\section{BSP模型的缺陷}

一个基于BSP程序同时具有水平和垂直两个方面的结构。从垂直上看,一个BSP程序由一系列串行的超步组成,每个超步又分为3个过程[12]:

(1)各个处理机进行本地局部计算。

(2)各个处理机利用本地内存中的信息完成局部的计算工作,在这一阶段处  理机发出远程内存读取和消息通信等工作。

(3)所有处理机进行全局路障同步,本次超步的通信操作在路障同步之后变为有效。


\section{计算模型改进}

BSP模型在分布式图处理系统中被广泛采用,主要是因为BSP模型非常适合分布式的图计算环境。首先,BSP模型具有全局的数据通信网络。BSP中的各个处理单元和通过它和其他处理单元进行通信或内存的存取,这和大多数基于分布式内存或消息传递的并行模型是相同的。不同的是BSP的通信是一个全局的概念,不是点对点的通信,能够很好的解决图计算过程中数据的随机访问问题。其次,BSP模型具有全局的路障同步机制。路障同步的引入则能保证图处理过程的完整性,提高整个图处理的鲁棒性和可靠性。因此,很多基于分布式内存的大规模图处理统都构建在BSP模型的基础上,例如Pregel、GPS等。

鉴于当前的计算机以多核为主,单个计算机的计算能力、并发处理能力已经有了很大的提高,将BSP模型从分布式的环境中迁移到这样的多核的计算机上不仅可以充分利用多核的计算资源,还能够降低图计算的处理成本。同时,考虑到单机图处理系统与分布式图处理系统的兼容性,本文首先对BSP模型中存在的问题进行了分析,然后针对这些问题提出New BSP模型,并在New BSP模型的基础上实现一个便捷、高效、可靠的单机图处理系统GPSA。

\vspace{\baselineskip}
\begin{figure}[htbp]
\centering
\begin{minipage}{0.4\textwidth}
\centering
\includegraphics[width=\textwidth]{myfigures/sequentialbsp_new}
\caption{传统BSP模型}\label{fig:traBSP}
\end{minipage}
\begin{minipage}{0.4\textwidth}
\centering
\includegraphics[width=\textwidth]{myfigures/computemodel}
\caption{改进后的BSP模型}\label{fig:newBSP}
\end{minipage}
\vspace{\baselineskip}
\end{figure}


\subsection{传统BSP模型的缺陷}
由于BSP模型的同步路障机制的存在,那么在整个BSP模型的垂直方向上,影响最终效率的关键就是最晚完成的任务,如图\ref{fig:bsp}所示。在基于传统BSP计算模型的图处理系统中,如图\ref{fig:traBSP}所示,图的处理主要分为两个阶段:以顶点为中心的计算(Compute)过程和消息的分发(Dispatch)过程。这两个过程按照严格的串行方式执行的,消息分发过程对计算过程存在数据依赖,形成强耦合。这种串行执行的方式,进一步延长了垂直方向上的任务完成的时间,降低的执行效率。在以顶点为中心的图计算中,消息时顶点之间的主要数据交换,顶点调用用户提供的计算函数对数据进行处理。消息包含计算所需的全部数据,所以相对于计算过程而言消息的具体分发过程是可以透明的。由于消息的生成和处理分布在两个相邻的超级步中,在该超级步结束之前,需要缓存大量的消息,并且在下一个超级步开始之前,这些消息不会被处理掉,增加额外的IO开销。
目前,基于BSP计算模型的图处理系统多任务并行处理的方法主要是采用多线程的技术。而线程是操作系统的基本调度单位,在操作系统中利用多线程在并发度上就会受到很大的限制。另外,由于需要保存大量的数据,当线程处理IO操作的时候,线程的调度会引发上下文的频繁切换,会对处理效率造成影响。

\subsection{Actor-BSP模型}

消息的分发过程和计算过程之间的主要依赖关系是消息,那么消息分发过程是消息的生产者,计算过程则是消息的消费者,两者之间以生产者和消费者的模式进行共存,从而将两者从严格串行的模式中解耦出来,如图\ref{fig:newBSP}所示。在改进后的BSP模型中,消息分发过程和计算过程位于两个单独的执行流程中。消息分发过程主要负责消息的生成和转发。计算过程侦听消息,当消息达到,计算过程则负责对消息进行处理及数据更新。
在语义上计算过程和消息分发过程是相互独立的轻量级调度单位。现有的BSP的实现在语义封装上往往是以顶点为中心,线程负责处理顶点。所以,在实现New BSP的模型中,采用轻量级、能够异步处理消息的并发模型成为关键。Actor并发模型相比于线程而言更加轻量级,有着更好的并发度。同时,Actor在语义上与Vertex的语义较为接近,兼顾了线程的调度执行和以顶点为中心的特点。

\subsection{Actor-BSP模型的优势}

Actor-BSP模型是在执行过程上完全异步计算的模型,计算过程与消息分发过程想分离,使得两个过程可以在一定程度上并行执行,充分利用多核的优势。

首先,在Actor-BSP模型中,Actor分为两种类型:负责消息分发过程的dispatching actor和负责计算过程的computing actor。得益于dispatching actor和computing actor之间的松耦合设计,两者之间的映射关系变得相当的灵活,缩短BSP模型在垂直方向上的执行流程,提高效率。

其次,dispatching actor 和 computing actor 组成了生产者和消费者模型,当消息到达computing actor之后,该actor会被调度执行,处理消息,从而无需保存这些消息,避免将消息进行缓存的IO开销。

\section{数据组织}

单机的图处理系统同时具有计算密集型和IO密集型两个特点,合理的数据组织方式对整个系统的性能具有重要的影响。本节首先分析在Actor-BSP模型中数据的访问行为,然后根据数据的行为对GPSA的数据的组织方式进行说明和解释。

\subsection{数据访问行为}
在Actor-BSP模型中,Actor替换顶点成为整个计算展开的中心对象,之前顶点之间的消息的传递转变为Actor对象之间的消息通讯,从而导致新模型在数据读写方式上与传统的BSP模型不同。本小节从消息、顶点以及边的角度出发结合Actor-BSP模型展示其数据读写方式的行为特征。

首先,消息无需缓存。由于Actor-BSP模型中,由于生产者和消费者的组合,computing actor监听消息到达事件,一旦事件到达消息就可以被及时处理。新模型无需为下一个超级步保存大量的消息,减少消息缓存的IO操作。

其次,只有dispatching actor才会访问边。由于dispatching actor主要任务就是生产消息,而消息的产生和发送都与边有着紧密的联系。但是,computing actor不关心边的状态,只关心消息到达的事件以及如何处理消息与更新数据。

再次,顶点状态数据信息需要常驻内存来支持随机访问。在以顶点为中心的模型中,顶点是有状态信息的,当顶点开始对消息进行处理的时候,顶点自身的状态会包含在自己的数据域,并且参与到计算中。无状态的Actor不会持有特定消息所有者的顶点状态信息。如果computing actor接收到消息,它需要获取消息所有者的状态信息,然后调用用户提供的函数对消息进行处理。然而无状态的Actor并不记录消息到来的顺序及消息所有者的任何信息,所以无状态的Actor在处理消息的时候需要能够随机获取消息所有者的状态信息。

最后,双份状态信息同时参与计算。在传统模型中,顶点的状态信息需要额外保存一份来判断状态是否发生更新。而在Actor-BSP模型中,顶点的状态信息的双份保存主要是因为dispatching actor和computing actor对状态信息读取的目的不同。dispatching actor获取顶点的状态信息主要是产生消息,而computing actor获取状态信息主要是用于处理消息,并且替换更新的状态信息。所以两者在获取的状态信息是不一致的。dispatching actor获取的状态信息主要是来自初始化或者上一个超级步的结果,而computing actor获取的状态信息却总是当前超步的最新结果。



\subsection{磁盘IO}
GraphChi和X-stream需要将大量的数据写入到磁盘中,因此对提高IO的性能有着非常高的要求。所以,Graphchi不遗余力的设计平行滑动窗口(PSW),并藉此实现异步的计算模型,并且为了实现PSW,GraphChi需要做非常多的预处理来满足要求。而X- stream也是从避免随机访问的角度出发采用顺序访问的Scatter和Gather两个阶段对图进行处理,同时采用异步IO的方式来获得更好的性能。

然而,在新的计算模型中,数据的访问方式发生较大的变化。不需要缓存消息意味着不需要为缓存大量的中间消息设计复杂的磁盘IO功能。边与顶点的分离,使得要处理的数据量减少很多。相比较整个庞大的图而言,顶点的状态信息就显得适合常驻内存进行处理来获取更好的性能。由于不需要向磁盘写入大量的数据,出于容错性的考虑,我们采用内存映射的方式对顶点的状态信息进行组织。而对于庞大数量的边而言,松散的组织结构可以省去一些耗时较长的预处理,同时满足顺序访问的特点,避免复杂的IO优化操作。

\subsection{数据组织设计}
图的数据主要分为两个部分,顶点的状态信息和边。其中,顶点的状态信息以二进制的格式存储在内存映射文件当中。边则以行压缩(Compressed Sparse Row)格式保存在磁盘上。

由于顶点的状态信息需要保存两份以供两种actor访问,所以在顶点状态信息的内存映射文件中,同一个顶点的状态信息以相邻的方式保存两份。通过这种简单的方式,可以快速的获取顶点\textit{V}的状态信息的值,即$|V| * sizeof(Val)*2$。当该内存映射文件被映射到内存中后,整个访问的过程就像在操作一个数组一样方便。
The CSR format stores vertices and edges in separate arrays, with the indices into these arrays corresponding to the identifier for the vertex or edge, respectively. The edge array is sorted by the source of each edge, but contains only the targets for the edges. The vertex array stores offsets into the edge array, providing the offset of the first edge outgoing from each vertex. Iteration over the out-edges for the ith vertex in the graph is achieved by visiting edge_array[vertex_array[i]], edge_array[vertex_array[i]+1], ..., edge_array[vertex_array[i+1]]. This format minimizes memory use to O(n + m), where n and m are the number of vertices and edges, respectively. The constants multiplied by n and m are based on the size of the integers needed to represent indices into the edge and vertex arrays, respectively, which can be controlled using the template parameters. The Directed template parameter controls whether one edge direction (the default) or both directions are stored. A directed CSR graph has Directed = directedS and a bidirectional CSR graph (with a limited set of constructors) has Directed = bidirectionalS.
CSR格式将顶点\textit{id}和边分别用两个数组进行保存,其中边数组保存边的目的顶点\textit{id}并按照出发顶点\textit{id}排序就,而在顶点数组中顶点的\textit{id}就是数组的索引,每个顶点保存第一个出边在在边数组中的索引,如图所示。


\begin{figure}[htbp]

 \begin{minipage}[]{0.5\textwidth}
 \centering
     \includegraphics[width=0.8\textwidth,angle=0,scale=1.2]{myfigures/csrdataformat.eps}
\end{minipage}
    \caption{\textbf{CSR example}  }
	\textit{(a) a example graph (b) a csrfile content without vertex degree (c) csrfile content with vertex degree}
    \label{figure:csrexample}

\end{figure}

\subsubsection{图结构组织}

\subsubsection{消息格式}


\section{消息分发}
\subsection{}
\subsection{分发策略}

\section{数据更新}


\section{GPSA实现}
\subsection{}

另外,在以顶点为中心的计算模型中,只有顶点的状态发生改变之后,才会发送消息,所以需要额外保存一份之前的状态数据来与当前计算过程完成之后的状态进行比较以判别顶点状态是否发生了改变。但是,在传统的BSP模型中,这两份状态数据,之前的状态数据直到下一个超步来临之前并不会被覆盖,并且这部分数据所占用的空间被白白浪费。

从BSP计算模型的垂直方向上看,它是由一些列串行的超步组成,而在一个超步的内部,图处理的两个过程也是串行处理。但是,在超级步S中
