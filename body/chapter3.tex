% !Mode:: "TeX:UTF-8"

\chapter{多核图计算}

\section{多核图计算简介}

图是一种复杂的抽象数据结构,能够表示不同物件之间的关系。图论作为数学领域中的一个重要分支已经有数百年的历史。人们发现图的许多重要特性,发明了许多重要的算法,其中许多困难问题的研究仍然十分活跃。在解决现实问题时,其他数据结构,例如线性表、树等,都具有明显的条件限制,而图结构中任意两个数据元素间均可相关联。在结构和语义方面,图较之线性表和树更为复杂,但是也更加具有表现能力。因此,现实世界中的数据往往都是以图的结构进行保存,而现实世界的数据量又非常巨大,常规的处理方法难以满足要求,因此需要进行大规模的图计算的研究。

目前,针对大规模图处理已经存在一些分布式的解决方案,例如,Pregel、GPS、Mizan等,这些分布式系统采用Vertex-Centric模型和BSP计算模型,顶点之间互通消息,最终达到顶点的收敛状态。但是分布式的图处理系统依然存在着许多悬而未决的难题。因此,对于分布式图处理系统而言,如何解决负载均衡、消息通讯问题,则成为其突破性能瓶颈的关键。

随着计算机多核处理器的出现和内存不断增大,单台计算机的计算性能实际上已经有着很大的提升。
在过去40多年时间里,计算机性能一直遵循着摩尔定律,集成电路上可容纳的晶体管数目,约每隔18个月便会增加一倍,而集成电路的性能(计算能力)也将提升一倍。近年来,集成电路的集成程度已经非常高,芯片上元件的几何尺寸不可能无限制的缩小下去,摩尔定律面临挑战,遭遇瓶颈。另外,仅仅提高单核芯片的速度会产生过多的热量并且无法带来相应的性能改善。然而,人们对于电脑的要求不断提高,迫使处理器向高性能的方向发展。如果多一颗同一性能的处理器,理论上处理能力是原来的两倍。于是,为了进一步提高性能,就需要更多的处理器,将多个处理器置入单一芯片中,构成多核心处理器。于是,有相关学者和研究人员提出基于单机共享内存的图处理系统,例如GraphChi、X-Stream等。实验证明,这些单机上图处理系统经过严谨而合理的设计不仅具有高效便捷的优势,还能大大的降低开发成本。但是,随着网络的发展,数据量的海量递增趋势已经越来越明显,现有的一些单机系统在扩展性上很难满足这样的挑战。目前应用于企业级的图处理系统大部分仍然基于BSP模型的分布式系统,而现有单机系统往往从系统IO优化处着手,两者之间很难互相兼容。另外,局限于目前多核的并发编程理论并不成熟,造成单机环境下多核计算机的计算能力并没有被充分利用。传统上的单机多核并行软件开发往往是基于共享内存的,共享内存就很容易引起竞争条件,为保证程序的准确性和一致性,就必须对可能引起竞争的变量或者语句进行加锁保证互斥或者进行同步。此外,从硬件层面考虑,多核虽然提高了计算性能,但是随着核数的增多,核与核之间也会出现消息通讯,内存数据一致性等问题.

常见的图算法有遍历、最短路径、最大连通分量等。对于规模较小可以完全载入内存的图,单线程的处理方式需要顺序依次对顶点进行遍历访问即可。但是,当图的规模巨大时,这种单线程的执行方式就难以满足需求。大规模图的处理技术同时具有IO密集型和计算密集型两种特点。IO密集型是因为大规模图对象数据量庞大,无法一次性全部载入内存进行计算,同时还会涉及大量的中间结果的读写。而计算密集型则是因为基于大规模图对象的应用需要对整个图进行分析计算,涉及大量的迭代和计算。因此,传统的单线程的开发方式不仅能造成IO阻塞降低图处理的效率,还无法充分发挥多核计算机的有事,造成计算资源的浪费。因此,对于大规模图对象的处理一般采用并发的方式来提高图处理的效率。

为了能够更加有效的利用硬件所提供的性能,传统的应用开发方式现在已经不适用了。以往的应用开发方式在大部分情况下所面对的都是只有一个单独的处理器,也就是意味着顺序执行的单线程应用。如今,多核计算机逐渐成为主流配置,并且价格也在不断地降低。而要处分发挥多核计算机的计算能力,最简单的办法就是利用并发,编写能够在多个核心上运行的任务,并且任务之间可以通过共享数据的方式进行通信协同工作,也就是并发程序。

为适应当今社会多核的迅猛发展,许多并发模型应运而生。在JVM平台上,已经存在的并发模型有传统同步模型、软件事务内存模型和角色模型三种。目前,实现并发程序有许多方式,以操作系统的支持,可以用进程,或是线程。以编程语言的支持,在JVM平台上可以使用多线程,JDK并发模型、软件事务内存、基于角色的并发模型。


\section{多线程在图计算中的应用}



首先,大规模图一般情况无法完全载入内存,需要分批依次载入并进行计算处理。这样图的处理就会涉及大量的IO读写操作,单线程程序会引起IO阻塞,不仅降低图的处理效率。其次,对于操作系统而言,线程是基本的调度单位,在双处理器的计算机上,单线程的程序只能利用一半的CPU资源,造成计算资源的浪费。而如果采用多线程的方式,




由于基本的调度单位是线程,因此如果在程序中只有一个线程,那么最多同时只能在一个处理器上运行。在双处理器系统上,单线程的程序只能使用一半的CPU资源,而在拥有100个处理器的系统上,将有99$\%$的资源无法使用。如果仍然按照传统的开发方式,多核的优势就无法发挥,对计算资源造成浪费。
另一方面,多线程程序可以同时在多个处理器上执行。如果设计正确,多线程程序可以通过提高处理器资源的利用率来提升系统吞吐率。使用多个线程还有助于在单处理器系统上获得更高的的吞吐率。如果程序是单线程的,那么当程序等待某个同步I/O操作完成时,处理器将处于空闲状态。而在多线程程序中,如果一个线程在等待I/O操作完成,另一个线程可以继续运行,使程序能够在I/O阻塞期间继续运行。




\section{Actor模型简介}

角色模型是一种不同的并发进程建模方式。与通过共享内存与锁交互的线程不同,角色模型利用了 “角色” 概念,使用邮箱来传递异步消息。在这里,邮箱类似于实际生活中的邮箱,消息可以存储并供其他角色检索,以便处理。邮箱有效地将各个进程彼此分开,而不用共享内存中的变量。

角色充当着独立且完全不同的实体,不会共享内存来进行通信。实际上,角色仅能通过邮箱通信。角色模型中没有锁和同步块,所以不会出现由它们引发的问题,比如死锁、严重的丢失更新问题。而且,角色能够并发工作,而不是采用某种顺序方式。因此,角色更加安全(不需要锁和同步),角色模型本身能够处理协调问题。在本质上,角色模型使并发编程更加简单了。

角色模型并不是一个新概念,它已经存在很长时间了。一些语言(比如 Erlang 和 Scala)的并发模型就是基于角色的,而不是基于线程。实际上,Erlang 在企业环境中的成功(Erlang 由 Ericsson 创建,在电信领域有着悠久的历史)无疑使角色模型变得更加流行,曝光率更高,而且这也使它成为了其他语言的一种可行的选择。Erlang 是角色模型更安全的并发编程方法的一个杰出示例。


\section{Kilim简介}

Kilim是一个使用JAVA编写的并发库,融入了角色模型的概念。Kilim由一个称为Weaver的后期进程来实现,该进程转换类的字节码。包含 Pausable throws 字句的方法在运行时由一个调度程序处理,该调度程序包含在 Kilim 库中。该调度程序处理有限数量的内核线程。可以利用此工具来处理更多的轻量型线程,这可以最大限度地提高上下文切换和启动的速度,并且每个线程的堆栈都是自动管理的。但是,Kilim设计之初并没有考虑到大量的协程之间的消息通讯,所以Kilim在面对频繁大量的协程通讯显得力不从心。

Disruptor[31]是一个开源的应用在Java平台上的低延迟高吞吐量的并发框架。LMAX是一种新型零售金融交易平台,Disruptor是LMAX的一个业务逻辑处理器,它能够在一个线程里每秒处理6百万订单,它完全运行在内存中,使用事件源驱动方式。在Disruptor框架中避免锁机制,添加缓存行填充来消除伪共享,以此提高并发性。但是,Disruptor的框架适用于线程,并且局限于特定平台。
因此,我们的工作主要围绕两个方面进行:1)、高并发编程的以顶点为中心的单机模型;2)、单机多核环境下高效的消息传递机制。
