% !Mode:: "TeX:UTF-8"

\chapter{多核图计算}

\section{多核计算简介}
在过去40多年时间里,计算机性能一直遵循着摩尔定律,集成电路上可容纳的晶体管数目,约每隔18个月便会增加一倍,而集成电路的性能(计算能力)也将提升一倍。近年来,集成电路的集成程度已经非常高,芯片上元件的几何尺寸不可能无限制的缩小下去,摩尔定律面临挑战,遭遇瓶颈。另外,仅仅提高单核芯片的速度会产生过多的热量并且无法带来相应的性能改善。然而,人们对于电脑的要求不断提高,迫使处理器向高性能的方向发展。如果多一颗同一性能的处理器,理论上处理能力是原来的两倍。于是,为了进一步提高性能,就需要更多的处理器,将多个处理器置入单一芯片中,构成多核心处理器。

为了能够更加有效的利用硬件所提供的性能,传统的应用开发方式现在已经不适用了。以往的应用开发方式在大部分情况下所面对的都是只有一个单独的处理器,也就是意味着顺序执行的单线程应用。如今,多核计算机逐渐成为主流配置,并且价格也在不断地降低。而要处分发挥多核计算机的计算能力,最简单的办法就是利用并发,编写能够在多个核心上运行的任务,并且任务之间可以通过共享数据的方式进行通信协同工作。并发计算是一种程序运算的特性,可以被视为是并行运算的进一步抽象,它包涵了时间片这种可以被用来实现虚拟并行运算(pseudoparallelism)的技术,因此在实际的物理运作中,计算过程可能是并行,或非并行的。并发计算,简单来说,就是将一个计算任务,分区成几个小的部份,让它们同时被计算,之后再汇整计算结果,以完成任务。它跟并行计算(Parallel computing),有重叠之处,在概念上不同,但常会让人混淆。

目前,实现并发程序有许多方式,以操作系统的支持,可以用进程,或是线程。以编程语言的支持,在JVM平台上可以使用多线程,JDK并发模型、软件事务内存、基于角色的并发模型。本文主要着重探讨在JVM平台上的各种并发模型。


\section{线程}

由于基本的调度单位是线程,因此如果在程序中只有一个线程,那么最多同时只能在一个处理器上运行。在双处理器系统上,单线程的程序只能使用一半的CPU资源,而在拥有100个处理器的系统上,将有99$\%$的资源无法使用。如果仍然按照传统的开发方式,多核的优势就无法发挥,对计算资源造成浪费。
另一方面,多线程程序可以同时在多个处理器上执行。如果设计正确,多线程程序可以通过提高处理器资源的利用率来提升系统吞吐率。使用多个线程还有助于在单处理器系统上获得更高的的吞吐率。如果程序是单线程的,那么当程序等待某个同步I/O操作完成时,处理器将处于空闲状态。而在多线程程序中,如果一个线程在等待I/O操作完成,另一个线程可以继续运行,使程序能够在I/O阻塞期间继续运行。




\section{Actor模型简介}
角色模型是一种不同的并发进程建模方式。与通过共享内存与锁交互的线程不同,角色模型利用了 “角色” 概念,使用邮箱来传递异步消息。在这里,邮箱类似于实际生活中的邮箱,消息可以存储并供其他角色检索,以便处理。邮箱有效地将各个进程彼此分开,而不用共享内存中的变量。

角色充当着独立且完全不同的实体,不会共享内存来进行通信。实际上,角色仅能通过邮箱通信。角色模型中没有锁和同步块,所以不会出现由它们引发的问题,比如死锁、严重的丢失更新问题。而且,角色能够并发工作,而不是采用某种顺序方式。因此,角色更加安全(不需要锁和同步),角色模型本身能够处理协调问题。在本质上,角色模型使并发编程更加简单了。

角色模型并不是一个新概念,它已经存在很长时间了。一些语言(比如 Erlang 和 Scala)的并发模型就是基于角色的,而不是基于线程。实际上,Erlang 在企业环境中的成功(Erlang 由 Ericsson 创建,在电信领域有着悠久的历史)无疑使角色模型变得更加流行,曝光率更高,而且这也使它成为了其他语言的一种可行的选择。Erlang 是角色模型更安全的并发编程方法的一个杰出示例。


\section{Kilim简介}




