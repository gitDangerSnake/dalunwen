% !Mode:: "TeX:UTF-8"

\chapter{多核计算}

\section{多核计算简介}
在过去40多年时间里,计算机性能一直遵循着摩尔定律,集成电路上可容纳的晶体管数目,约每隔18个月便会增加一倍,而集成电路的性能(计算能力)也将提升一倍。近年来,集成电路的集成程度已经非常高,芯片上元件的几何尺寸不可能无限制的缩小下去,摩尔定律面临挑战,遭遇瓶颈。另外,仅仅提高单核芯片的速度会产生过多的热量并且无法带来相应的性能改善。然而,人们对于电脑的要求不断提高,迫使处理器向高性能的方向发展。如果多一颗同一性能的处理器,理论上处理能力是原来的两倍。于是,为了进一步提高性能,就需要更多的处理器,将多个处理器置入单一芯片中,构成多核心处理器。

为了能够更加有效的利用硬件所提供的性能,传统的应用开发方式现在已经不适用了。以往的应用开发方式在大部分情况下所面对的都是只有一个单独的处理器,也就是意味着顺序执行的单线程应用。如今,多核计算机逐渐成为主流配置,并且价格也在不断地降低。而要处分发挥多核计算机的计算能力,最简单的办法就是利用并发,编写能够在多个核心上运行的任务,并且任务之间可以通过共享数据的方式进行通信协同工作。并发计算是一种程序运算的特性,可以被视为是并行运算的进一步抽象,它包涵了时间片这种可以被用来实现虚拟并行运算(pseudoparallelism)的技术,因此在实际的物理运作中,计算过程可能是并行,或非并行的。并发计算,简单来说,就是将一个计算任务,分区成几个小的部份,让它们同时被计算,之后再汇整计算结果,以完成任务。它跟并行计算(Parallel computing),有重叠之处,在概念上不同,但常会让人混淆。

目前,实现并发程序有许多方式,以操作系统的支持,可以用进程,或是线程。以编程语言的支持,在JVM平台上可以使用多线程,JDK并发模型、软件事务内存、基于角色的并发模型。本文主要着重探讨在JVM平台上的各种并发模型。


\section{线程}

由于基本的调度单位是线程,因此如果在程序中只有一个线程,那么最多同时只能在一个处理器上运行。在双处理器系统上,单线程的程序只能使用一半的CPU资源,而在拥有100个处理器的系统上,将有99$\%$的资源无法使用。如果仍然按照传统的开发方式,多核的优势就无法发挥,对计算资源造成浪费。
另一方面,多线程程序可以同时在多个处理器上执行。如果设计正确,多线程程序可以通过提高处理器资源的利用率来提升系统吞吐率。使用多个线程还有助于在单处理器系统上获得更高的的吞吐率。如果程序是单线程的,那么当程序等待某个同步I/O操作完成时,处理器将处于空闲状态。而在多线程程序中,如果一个线程在等待I/O操作完成,另一个线程可以继续运行,使程序能够在I/O阻塞期间继续运行。




\section{Actor模型简介}

\section{Kilim简介}




