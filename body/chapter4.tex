% !Mode:: "TeX:UTF-8"

\chapter{系统测试与分析}

为了证实GPSA系统的实际性能,本章对该系统进行了测试和分析。首先从硬件、软件以及用作测试的数据集三方面介绍了测试环境;接着介绍在GPSA中PageRank、连通分量和广度搜索三个应用的实现,测试它们在不同数据集上的运行性能,并将结果与GraphChi和X-Stream进行对比分析;最后,从对单机多核的利用率角度对系统进行测试和分析。


\section{实验环境}

\subsection{硬件环境}
本文的系统与测试都是运行在相同的计算机上。该计算机CPU为32核,主频1.8GHZ的Intel i7 cores, 16GB内存,1TB硬盘,转速7200rpm。
\subsection{软件环境}
本文的操作系统为Ubuntu 12.04LTS,JDK7,Kilim。另外,本文用来对比的单机图处理系统分别为GraphChi(0.2.6 C++)与X-Stream。

\subsection{测试数据集}
为了能够全面的了解GPSA的运行性能,所以我们选择四个大小不同的数据集:Google数据集、soc-pokec数据集、soc-liveJournal数据集以及twitter-2010数据集。四个数据集的大小情况如下表所示:
\renewcommand\arraystretch{1.5}%控制行距
\begin{table}[!h]
\caption{基准测试集参数}\label{tab:bench}
\vspace{0.5em}
\centering
\begin{tabular}{l*{1}{C{2cm}}*{1}{L{6.2cm}}}
\toprule
Name         & Nodes & Edges\\
\midrule
google      & 875,713  & 5,105,039\\
soc-pokec      & 1,632,803 & 30,622,564\\
soc-liveJournal      & 4,847,571 & 68,993,773\\
twitter-2010      & 41,652,230 & 1,468,365,182\\

\bottomrule
\end{tabular}
\vspace{\baselineskip}
\end{table}
\renewcommand\arraystretch{1}


\section{性能测试}



\section{多核利用率测试}
\section{小结}
\subsection{..}
\subsection{..}
\subsection{..}
\subsection{..}






